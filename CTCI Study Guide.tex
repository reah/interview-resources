\documentclass[10pt]{article}
\usepackage[margin=2cm]{geometry}
\usepackage[fleqn]{amsmath}
\usepackage{bm,amssymb,mathrsfs}
\usepackage{scrextend}


\setlength{\mathindent}{0.0cm}

\def\yields{\hskip 5 pt $\to$ \hskip 5 pt}
\newcommand{\bs}[1]{\pmb{#1}}
\newcommand{\parfrac}[2]{\frac{\partial #1}{\partial #2}}

\begin{document}

\setcounter{secnumdepth}{-1}

\title{Cracking the Coding Interview -- STUDY GUIDE}
\author{by {\bf reah miyara} }%\& {\bf david scherban}}
\date{march 28, 2016}
\maketitle

\section{Programming Paradigms}\smallskip

\paragraph{Object Oriented}\ \\
In object oriented languages {\bf data and methods of manipulating the data are kept as a single unit called an object}. The only way a user can access the data is via object's methods. Therefore the inner workings of an object may be changed without affecting any code that uses the object. 
\begin{addmargin}[3em]{2em}% 1em left, 2em right

{\indent \vspace{-10pt} \paragraph{Polymorphism}\ \\
 Providing or supplying a single interface to be used with entities of different types. }
\end{addmargin}

\paragraph{Declarative}\ \\
In declarative languages the computer is told {\bf what the problem is, not how to solve the problem} --the program is structured as a collection of properties to find in the expected result, not as a procedure to follow. It's a style of expressing the logic of a computation without describing its control flow. This is in {\bf contrast with imperative programming}, which implements algorithms in explicit steps. {\it Given a database or a set of rules, the computer tries to find a solution matching all the desired properties., e.g. SQL}

\paragraph{Imperative}\ \\
Imperative programming focuses on {\bf how a program operates}, consisting of commands for the computer to perform. This {\bf contrasts declarative programming}.

\paragraph{Functional}\ \\
Functional programming is a {\bf subset of declarative programming}. Programs are written using functions, {\it blocks of code intended to behave like mathematical functions}. Functional languages discourage changes in the value of variables through assignment.   

\noindent\makebox[\linewidth]{\rule{\paperwidth}{0.4pt}}


\paragraph{Properties of supremum and infimum}\ \\
Let $h$ be a given positive number and let $S$ be a set of real numbers.\\ 
(a) If $S$ has a supremum, then for some $x$ in $S$ we have $x > \sup S - h$.\\
(b) If $S$ has an infimum, then for some $x$ in $S$ we have $x < \inf S + h$.

\paragraph{Well-ordering principle}\ \\
Every nonempty set of positive integers contains a smallest member.

\paragraph{Triangle inequality}\ \\
For arbitrary real numbers $x$ and $y$, 
$|x + y| \leq |x| + |y|.$
More generally, for arbitrary real numbers $a_1$, $a_2$, \ldots, $a_n$, we have
$\left|\sum_{k=1}^n a_k\right| \leq \sum_{k=1}^n |a_k|.$

\paragraph{The Cauchy-Schwarz inequality}\ \\
If $a_1, \ldots, a_n$ and $b_1, \ldots, b_n$ are arbitrary real numbers, we have
$\left( \sum_{k=1}^n a_k b_k \right)^2 \leq \left( \sum_{k=1}^n a_k^2 \right) \left( \sum_{k=1}^n b_k^2 \right)$.
The equality sign holds if and only if there is a real number $x$ such that $a_k x + b_k = 0$
for each $k = 1, 2, \ldots, n$.



\bigskip\bigskip
\section{Complex Field}\smallskip

\paragraph{Field properties}\ \\
$(a,b)=(c,d)$ means $a=c$ and $b=d$\\
$(a,b)+(c,d)=(a+c,b+d)$\\
$(a,b)(c,d)=(ac-bd,ad+bc)$
$x+y=y+x$\\
$x+(y+z)=(z+y)+z$\\
$x(y+z)=xy+xz$\\
$e^{z+2n\pi i}=e^z$

\begin{addmargin}[3em]{2em}% 1em left, 2em right

{\indent \paragraph{Polar coordinates}\ \\
$x = r \cos \theta$ \hskip 10 pt $y = r \sin \theta$\\
$r$ is the modulus or absolute value of $(x,y)$, equal to $\sqrt{x^2+y^2}$.\\
$\theta$ is the angle between $(x,y)$ and the x-axis, and is called the argument of $(x,y)$,
or the principal argument if $-\pi < \theta \leq \pi$.\\
Polar form of $z$: Every complex number $z \neq 0$ can be expressed as $z=re^{i\theta}$.}

\end{addmargin}


\paragraph{Complex exponential}\ \\
If $z=(x,y)$, then $e^z = e^x(\cos y + i \sin y)$\\
$e^ae^b=e^{a+b}$

\paragraph{Derivatives and integrals}\ \\
If $f=u+iv$, then $f'(x)=u'(x)+iv'(x)$\\
$\int_a^bf(x)\,dx=\int_a^bu(x)\,dx+i\int_a^bv(x)\,dx$\\
$(e^{tx})'=te^{tx}$\\
$\int e^{tx}\,dx=\frac{e^{tx}}{t}$



\end{document}































