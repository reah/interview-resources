\documentclass[10pt]{article}
\usepackage[margin=2cm]{geometry}
\usepackage[fleqn]{amsmath}
\usepackage[T1]{fontenc}
\usepackage{bm,amssymb,mathrsfs}
\usepackage{scrextend}
\usepackage{listings}
\usepackage{inconsolata}
\usepackage{color}
\usepackage{xcolor}
\usepackage{collectbox}
%\usepackage{courier}

\definecolor{pblue}{rgb}{0.13,0.13,1}
\definecolor{pgreen}{rgb}{0,0.5,0}
\definecolor{pred}{rgb}{0.9,0,0}
\definecolor{pgrey}{rgb}{0.83,0.83,0.83}
%\def\code#1{\colorbox{white}{\texttt{#1}}}

\newcommand{\code}[1]{%
    {\fcolorbox{black}{pgrey}{\texttt{#1}}}
}

\lstset{language=Java,
  showspaces=false,
  showtabs=false,
  breaklines=true,
  showstringspaces=false,
  breakatwhitespace=true,
  commentstyle=\color{pgreen},
  keywordstyle=\color{pblue},
  stringstyle=\color{pred},
  basicstyle=\ttfamily,
  moredelim=[il][\textcolor{pgrey}]{$$},
  moredelim=[is][\textcolor{pgrey}]{\%\%}{\%\%}
}

\newcommand{\mybox}{%
    \collectbox{%
        \setlength{\fboxsep}{1.5pt}%
        \fbox{\BOXCONTENT}%
    }%
}

\setlength{\mathindent}{0.0cm}

\def\yields{\hskip 5 pt $\to$ \hskip 5 pt}
\newcommand{\bs}[1]{\pmb{#1}}
\newcommand{\parfrac}[2]{\frac{\partial #1}{\partial #2}}

\begin{document}

\setcounter{secnumdepth}{-1}

\title{Cracking the Coding Interview -- STUDY GUIDE}
\author{by {\bf reah miyara} }
\date{\today}
\maketitle

\section{Programming Paradigms}\smallskip
\paragraph{Object Oriented}\ \\
In object oriented languages {\bf data and methods of manipulating the data are kept as a single unit called an object}. The only way a user can access the data is via object's methods. Therefore the inner workings of an object may be changed without affecting any code that uses the object. 
\begin{addmargin}[3em]{2em}% 1em left, 2em right

{\indent \vspace{-10pt} \paragraph{Polymorphism}\ \\
 Providing or supplying a single interface to be used with entities of different types. }
\end{addmargin}

\paragraph{Declarative}\ \\
In declarative languages the computer is told {\bf what the problem is, not how to solve the problem} --the program is structured as a collection of properties to find in the expected result, not as a procedure to follow. It's a style of expressing the logic of a computation without describing its control flow. This is in {\bf contrast with imperative programming}, which implements algorithms in explicit steps. {\it Given a database or a set of rules, the computer tries to find a solution matching all the desired properties., e.g. SQL}

\paragraph{Imperative}\ \\
Imperative programming focuses on {\bf how a program operates}, consisting of commands for the computer to perform. This {\bf contrasts declarative programming}.

\paragraph{Functional}\ \\
Functional programming is a {\bf subset of declarative programming}. Programs are written using functions, {\it blocks of code intended to behave like mathematical functions}. Functional languages discourage changes in the value of variables through assignment.   

\section{Arrays \& Strings}\smallskip
\begin{enumerate}

\item {\bf Determine if string has unique characters; without additional data structures.}

\begin{itemize}
\item split up characters from string \code{chars[] = str.toCharArray();}
\item get single character from string \code{char c = str.charAt(i);}
\item cast char to int, \code{int value = str.charAt(i);} , (this is because the primitive char datatype is a 16 bit unsigned integer)
\item shift 1 by the int casted from the char \&\& with a checker 0, if result isn't 0 then not unique, if it is checker |= result
\end{itemize}

\item {\bf Reverse a String}

\begin{itemize}
\item Strings are immutable so use StringBuffer, {\it (mutable, stored on the heap, each method is thread safe, synchronized)}, or StringBuilder, {\it (same methods as StringBuffer but not synchronized, hence not thread safe)}, and finalize the string using \code{StringBuilder.toString()}
\item method 1: copy backwards into StringBuffer linearly \\\code{for (int i = s.length()-1; i >= 0; i--) \{ sb.append(s.charAt(i)); \} return sb.toString();} 
\item method 2 (in place): turn string into char array \\\code{char[] reverse = s.toCharArray(); } and concurrently swap last and first chars until reach the middle \\
\code{for(int i = 0, j = s.length()-1; i < s.length()/2; i++, j--) \{}\\\code{char temp = reverse[i]; reverse[i] = reverse[j]; reverse[j] = temp;} \\\code{\} return new String(reverse);}
\item method 3: \code{return new StringBuilder(s).reverse().toString();}
\end{itemize}

\item {\bf Given 2 strings write a method to determine if one is a permutation of the other}

\begin{itemize}
\item check to see if lengths are the same \code{if(s1.length() != s2.length()) return false;}
\item method 1, sort O(nlogn): convert each string to an array of chars\\\code{char[] chars = s.toCharArray();} sort each in place \code{Arrays.sort(chars);} compare using  \\\code{Objects.equals(chars1, chars2);} which checks for \code{null} unlike \code{s1.equals(s2);}
\item initialize array of ints, if unicode, or hashmap otherwise to keep count of chars in s1. \\\code{int[] charCount = new int[128];} or \code{HashMap<Character, Integer> charCount = HashMap<C,I>();}
\item iterate over first string incrementing count for char \\\code{for(int i = 0; i < s1.length(); i++) \{}
\\\code{charCount[s1.charAt(i)]++;\}} or \\\code{if(map.get(s1.charAt(i) != null) \{ map.put(s1.charAt(i), map.get(s1.charAt(i)) + 1); \}}
\\\code{else \{ map.put(s1.charAt(i), 1); \}}
\item iterate over second string decrementing count for char, returning false if count falls below 0 
\\\code{for(int i = 0; i < s2.length(); i++) \{ if(--charCount[s2.charAt(i)] < 0) return false; \}} or  
\\\code{if(charCount.get(s2.charAt(i)) == null || (charCount.get(s2.charAt(i)) - 1 < 0)) \{ return false; \} }
\\\code{\} else \{ charCount.put(s2.charAt(i), charCount.get(s2.charAt(i)) - 1); \}}
\end{itemize}

\item {\bf Write a method to replace all spaces in a string with \lq \textnormal{\%20}\rq}
\begin{itemize} 
\item method 1, StringBuffer: create a StringBuffer object, convert String to char array, append each char that does not equal \lq\hspace{3pt}\rq\hspace{1pt}  to the StringBuffer otherwise append \lq{\textnormal\%20}\rq
\item method 2, using only primitive types: count the spaces in the String, calculate new size of char array needed to contain new string, copy each char if not  \lq\hspace{3pt}\rq\hspace{1pt}, otherwise \lq{\textnormal\%20}\rq by using a second iterator {\textnormal j}
\\\code{int newSize = s.length() + spaceCount * 2; char[] modified = new char[newSize];}
\item method 3, given char[] that can fit new string and true length of original string: calculate new size of string again by first counting spaces \code{int index = s.length + spaceCount * 2 - 1;} 
\\now in order to not overwrite original characters when replacing \lq\hspace{3pt}\rq\hspace{1pt}, insert each char backwards with the index calculated and true length given, finally returning new string
\begin{lstlisting}[frame=single]
for(int i = length; i >=0; i--) {
	if(s[i] == ' ') {
		s[index--] = '0';
		s[index--] = '2';
		s[index--] = '%';
	} else {
		s[index--] = s[i];
	}
}
return new String(s);
\end{lstlisting}\end{itemize}
\end{enumerate}

\section{General Java Knowledge}\smallskip

\paragraph{Unicode}\ \\
Java inherently supports unicode, encoding standard which contains 128 different characters, for char primitive types. 



\newpage
\noindent\makebox[\linewidth]{\rule{\paperwidth}{0.4pt}}

code input sample
\begin{lstlisting}[frame=single]
public class isUnique {
	public static boolean isUnique(String s) {
		// hello world
		HashMap <Character, Boolean>  charMap = new HashMap<Character, Boolean>();
	return true;
	}
}
\end{lstlisting}


\paragraph{Properties of supremum and infimum}\ \\
Let $h$ be a given positive number and let $S$ be a set of real numbers.\\ 
(a) If $S$ has a supremum, then for some $x$ in $S$ we have $x > \sup S - h$.\\
(b) If $S$ has an infimum, then for some $x$ in $S$ we have $x < \inf S + h$.

\paragraph{Well-ordering principle}\ \\
Every nonempty set of positive integers contains a smallest member.

\paragraph{Triangle inequality}\ \\
For arbitrary real numbers $x$ and $y$, 
$|x + y| \leq |x| + |y|.$
More generally, for arbitrary real numbers $a_1$, $a_2$, \ldots, $a_n$, we have
$\left|\sum_{k=1}^n a_k\right| \leq \sum_{k=1}^n |a_k|.$

\paragraph{The Cauchy-Schwarz inequality}\ \\
If $a_1, \ldots, a_n$ and $b_1, \ldots, b_n$ are arbitrary real numbers, we have
$\left( \sum_{k=1}^n a_k b_k \right)^2 \leq \left( \sum_{k=1}^n a_k^2 \right) \left( \sum_{k=1}^n b_k^2 \right)$.
The equality sign holds if and only if there is a real number $x$ such that $a_k x + b_k = 0$
for each $k = 1, 2, \ldots, n$.



\bigskip\bigskip
\section{Complex Field}\smallskip

\paragraph{Field properties}\ \\
$(a,b)=(c,d)$ means $a=c$ and $b=d$\\
$(a,b)+(c,d)=(a+c,b+d)$\\
$(a,b)(c,d)=(ac-bd,ad+bc)$
$x+y=y+x$\\
$x+(y+z)=(z+y)+z$\\
$x(y+z)=xy+xz$\\
$e^{z+2n\pi i}=e^z$

\begin{addmargin}[3em]{2em}% 1em left, 2em right

{\indent \paragraph{Polar coordinates}\ \\
$x = r \cos \theta$ \hskip 10 pt $y = r \sin \theta$\\
$r$ is the modulus or absolute value of $(x,y)$, equal to $\sqrt{x^2+y^2}$.\\
$\theta$ is the angle between $(x,y)$ and the x-axis, and is called the argument of $(x,y)$,
or the principal argument if $-\pi < \theta \leq \pi$.\\
Polar form of $z$: Every complex number $z \neq 0$ can be expressed as $z=re^{i\theta}$.}

\end{addmargin}


\paragraph{Complex exponential}\ \\
If $z=(x,y)$, then $e^z = e^x(\cos y + i \sin y)$\\
$e^ae^b=e^{a+b}$

\paragraph{Derivatives and integrals}\ \\
If $f=u+iv$, then $f'(x)=u'(x)+iv'(x)$\\
$\int_a^bf(x)\,dx=\int_a^bu(x)\,dx+i\int_a^bv(x)\,dx$\\
$(e^{tx})'=te^{tx}$\\
$\int e^{tx}\,dx=\frac{e^{tx}}{t}$



\end{document}































